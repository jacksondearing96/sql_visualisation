%Document FORMATTING [D]
\documentclass[12pt]{article}
\usepackage[utf8]{inputenc}
\usepackage[top=2.5cm,bottom=2cm,right=4cm, left=4cm,includefoot]{geometry}
\usepackage{rotating}
\usepackage{tikz}

%BIBLIOGRAPHY
\usepackage[hyphen]{url}
\urlstyle{same}

\usepackage{booktabs}
\usepackage{lipsum}
\usepackage{setspace}
\usepackage{amsmath}

\usepackage{tabularx}
\usepackage{array}
\usepackage{floatrow} %forces table caption to top
\floatsetup[table]{capposition=top}

\usepackage{graphicx} %for figures
\usepackage{caption}
\captionsetup{margin=10pt,font=small} 
\usepackage{subcaption}

\usepackage{array}
\newcolumntype{L}[1]{>{\raggedright\let\newline\\\arraybackslash\hspace{4pt}}m{#1}}
\newcolumntype{C}[1]{>{\centering\let\newline\\\arraybackslash\hspace{4pt}}m{#1}}
\newcolumntype{R}[1]{>{\raggedleft\let\newline\\\arraybackslash\hspace{4pt}}m{#1}}
\usepackage{floatrow}
% Table float box with bottom caption, box width adjusted to content
\newfloatcommand{capbtabbox}{table}[][\FBwidth]
\usepackage{lastpage}   
\usepackage{fancyhdr}
\pagestyle{fancy}
\fancyhead{}        %clears fancy head
\fancyfoot{}        %clears fancy foot
\rfoot{\thepage\ of \pageref{LastPage}}
\renewcommand{\headrulewidth}{0pt}
\usepackage{titlesec}
\titleformat*{\section}{\large\bfseries}
\titleformat*{\subsection}{\bfseries}


\begin{document}
%TITLE PAGE [D]
\begin{titlepage}
    \begin{center}
        \vspace*{100pt}
        \large{Faculty of Engineering, Computer\\ and Mathematical Sciences}\\
        \vspace*{15pt}
  
        {\large SIVT 1}\\
        [-5pt]
        \line(1,0){350}\\
        [15pt]
        {\huge \bfseries Initial Report }\\
        [1pt]
        \line(1,0){350}\\
        [5pt]
        {\large \bfseries Software Engineering \& Project}
        \\
        [10pt]
        {\Large 2020}\\
        \vspace*{50pt}
        \textsc{\normalsize Members:}\\
        [10pt]
    \centering
    \begin{minipage}{ .45\textwidth}
        \centering
        {\normalsize 
        \flushleft Colin Ross
        \flushleft Elliot Wheatland
        \flushleft Steve Monger 
       \flushleft Kristina Plavsa
       \flushleft Rhett Hull
        \flushleft Gabriel Piragibe
         \flushleft Jackson Dearing 
       \flushleft Brian Yip
       \flushleft Joshua Tatton\\}
    \end{minipage}%
    \begin{minipage}{0.45\textwidth}
        \centering
        {\normalsize 
        \flushright id
        \flushright id
        \flushright id
        \flushright a1723663
        \flushright id
        \flushright id
        \flushright a1686411
        \flushright id
        \flushright a1690417\\}
    
    \end{minipage}\\
    
    \vspace*{20pt}
        \textsc{\normalsize Submitted on:}\\
        [5pt]
        {$14^{th}$ August 2020}\\
    \end{center}
\end{titlepage}
%______________________________________
\cleardoublepage
\newgeometry{margin=2.54cm,left=2.0cm,right=2.0cm, includefoot}
%_______________________________________

\section{Project Vision}
%Brief (100-200 words) summary of the project vision in your own words

\section{Customer Q\&A}

\section{Users}

\section{Software Architecture}

\section{Tech Stack and Standards}

\section{Snapshot}

\iffalse
Title Page:
Title: Initial Report of Group <Group ID>
Name of project
List of team members with their a-numbers
Project Vision:
Brief (100-200 words) summary of the project vision in your own words
Customer Q&A:
This section should summarise the questions you asked during the kickoff meeting and the client's responses. You should starting writing down potential questions as soon as you got assigned to a team and project
This section should also contain a brief (100-200 words) reflection on how the kickoff meeting went, what you could have done differently, and what you've learned for the next meeting, maybe including potential follow-up questions
Users:
Describe the roles you identified after the kickoff meeting (roles in the sense of different kinds of users of your app)
You should list the roles and provide brief summaries of their responsibilities and potential actions (3-4 sentences)
Software Architecture:
Rough sketch of the potential architecture of your application
Can be structured UML, but can also be an ad hoc boxes-and-arrows diagrams with a legend
Provide a brief (100-200 words) justification for the architecture you chose
Tech Stack and Standards:
List the likely tech stack for your application (distinguish between back-end and front-end, programming languages and frameworks)
Agree on tools for communication and development (e.g., Slack, IDEs, etc.)
Agree on coding standards
Provide justifications for your choices
Group Meetings and Team Member Roles:
How frequently and for how long do you intend to (virtually) meet?
Please remember to time-box the meetings!
When will you schedule the sprint retrospective meetings?
Did you arrange for additional feedback channels with the customer (e.g., via Slack or email)?
Please name the Scrum Masters for each sprint (each team member can only be the Scrum Master for up to one sprint)
Snapshot:
Provide a snapshot before first sprint (see snapshot assignment)

\fi
\end{document}